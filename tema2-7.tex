% TODO añadir secciones y subsecciones. 2 horas de presentacion! viene bien estructurar
% TODO añadir resumen al final


\documentclass[hyperref={pdfpagelabels=false},tree-dvips]{beamer}
% By  using hyperref={pdfpagelabels=false} you get rid off:
%	Package hyperref Warning: Option `pdfpagelabels' is turned off
%	(hyperref)                because \thepage is undefined.
%	Hyperref stopped early
\usetheme{Warsaw}
%\setbeameroption{show notes}
%\setbeameroption{hide notes} % default
%\setbeameroption{show notes on second screen=left}
%\setbeameroption{show only notes} % for printing the notes only


\usepackage[spanish]{babel}
\selectlanguage{spanish}
\usepackage[utf8]{inputenc}

\usepackage{adjustbox}
\usepackage{graphicx}

\usepackage{xcolor}
\definecolor{green}{RGB}{0,170,0} % redefine green for visibility

\usepackage{bold-extra} % for having boldfaced monospaced font in listings
\usepackage{listings}
\lstdefinestyle{procesos}{
	language=C,
	emptylines=1,
	breaklines=true,
	basicstyle=\ttfamily\bfseries\color{black},
	moredelim=**[is][\color{red}]{@red@}{@},
	moredelim=**[is][\color{blue}]{@blue@}{@},
	moredelim=**[is][\color{green}]{@green@}{@},
}

\usepackage{tikz}
\usetikzlibrary{arrows,calc,shapes,decorations.pathreplacing}

\usepackage{bytefield}
	\newcommand{\colorbitbox}[3]{%
	\rlap{\bitbox{#2}{\color{#1}\rule{\width}{\height}}}%
	\bitbox{#2}{#3}}
	\definecolor{lightcyan}{rgb}	{0.80, 	0.90   , 1   }
	\definecolor{lightgreen}{rgb}	{0.50, 	1   , 0.50}
	\definecolor{lightred}{rgb}		{1   , 	0.70, 0.71}
	\definecolor{lightblue}{rgb}	{0   , 	0.90, 1   }



\title{Tema 2.7. Subprogramas. Traducción}
\author{Pedro Javier Rodríguez Rodrigo, Víctor Cuadrado Juan}
%Basado en las trasparencias de José Luis Sierra y Juan Antonio Recio.}
\date{\today}

\begin{document}


%%%%%%%%%%%%%%%%%%%%%%%%%%%%%%%%%%%%%%%%%%%%%%%%%%%%%%%%%%%%%%%%%%%%%%%%%%%%%%%%
\begin{frame}
\titlepage
\end{frame}
%%%%%%%%%%%%%%%%%%%%%%%%%%%%%%%%%%%%%%%%%%%%%%%%%%%%%%%%%%%%%%%%%%%%%%%%%%%%%%%%
\section{Organización de la memoria}
\begin{frame}[fragile]
\frametitle{Organización de la memoria}
\note{Talk no more than 1 minute.}

\begin{itemize}[<+->]% [<+->] makes it uncoverable
	\item Es posible acceder a los datos globales
	\item Desde cualquier registro de activación es necesario referir al registro de activación asociado con el bloque padre (que no tiene porque ser necesariamente el registro de activación anterior).
	\item Dos Posibles organizaciones:
		\begin{itemize}[<+->]
			\item Enlaces estáticos
			\item Displays
		\end{itemize}
\end{itemize}

\end{frame}
%%%%%%%%%%%%%%%%%%%%%%%%%%%%%%%%%%%%%%%%%%%%%%%%%%%%%%%%%%%%%%%%%%%%%%%%%%%%%%%%
\subsection{Enlaces estáticos}

\begin{frame}[fragile] % fragile: needed for verbatim/lstlistings
\frametitle{Enlaces estáticos: ejemplo}

\begin{columns}[T]
\column{.4\textwidth}
	\begin{lstlisting}[style=procesos]
	@red@proc proc1(){
	    x: num;
	    y: num;
	    proc1();
	}@
	@blue@proc proc2(){
	    w: bool;
	    proc1();
	}@
	@green@main(){
	    a: bool;
	    proc2();
	}@
	\end{lstlisting}
\column{.3\textwidth}
	\begin{tikzpicture}
	  \begin{scope}[every node/.style={draw, anchor=text, rectangle split,
	    rectangle split parts=14,minimum width=2cm}]
	    \node (R) at (2,4){
	    	\nodepart{two}proc2()
	    	\nodepart{five}proc1()
	    	\nodepart{nine}proc1()
	    	\nodepart{twelve}proc2()
	    	\nodepart{fourteen}proc1()
	    };
	  \end{scope}
	\end{tikzpicture}
\column{.3\textwidth}
	aqui la pila de registros
\end{columns}

\end{frame}
%%%%%%%%%%%%%%%%%%%%%%%%%%%%%%%%%%%%%%%%%%%%%%%%%%%%%%%%%%%%%%%%%%%%%%%%%%%%%%%%
\begin{frame}[fragile]
\frametitle{Enlaces estáticos}
\begin{itemize}%[<+->]% [<+->] makes it uncoverable
	\item En el registro de activación se incluye un enlace al registro de activación del bloque padre (enlace estático)
	\item La memoria se organiza en forma de pila de registros de activación, enlazados a través de los enlaces estáticos
\end{itemize}

Dibujito guapo

\end{frame}
%%%%%%%%%%%%%%%%%%%%%%%%%%%%%%%%%%%%%%%%%%%%%%%%%%%%%%%%%%%%%%%%%%%%%%%%%%%%%%%%
\begin{frame}[fragile]

\frametitle{Enlaces estáticos: problemas}

\uncover<1->{¿Qué problemas hay?}

\begin{enumerate}%[<+->]% [<+->] makes it uncoverable
	\item<2-> La recuperación del enlace de un identificador global supone seguir toda la cadena de enlaces estáticos. Si el identificador ha sido declarado \emph{k} niveles por encima, es necesario realizar \emph{k} indirecciones hasta llegar al correspondiente registro de activación.
	\item<3-> Hay que considerar la complejidad de generar código que gestione de manera adecuada los enlaces estáticos.
\end{enumerate}

\uncover<4->{Solución:\\
Almacenar los enlaces estáticos \emph{fuera} de los registros de activación. La estructura que los almacena se llama \textbf{display}.}

\end{frame}
%%%%%%%%%%%%%%%%%%%%%%%%%%%%%%%%%%%%%%%%%%%%%%%%%%%%%%%%%%%%%%%%%%%%%%%%%%%%%%%%
\subsection{Displays}

\begin{frame}[fragile]

% \frametitle{Enlaces estáticos: problemas}

% \uncover<1->{¿Qué problemas hay?}

% \begin{enumerate}%[<+->]% [<+->] makes it uncoverable
% 	\item<2-> La recuperación del enlace de un identificador global supone seguir toda la cadena de enlaces estáticos. Si el identificador ha sido declarado \emph{k} niveles por encima, es necesario realizar \emph{k} indirecciones hasta llegar al correspondiente registro de activación.
% 	\item<3-> Hay que considerar la complejidad de generar código que gestione de manera adecuada los enlaces estáticos.
% \end{enumerate}

% \uncover<4->{Solución:\\
% Almacenar los enlaces estáticos \emph{fuera} de los registros de activación. La estructura que los almacena se llama \textbf{display}.}

\end{frame}
%%%%%%%%%%%%%%%%%%%%%%%%%%%%%%%%%%%%%%%%%%%%%%%%%%%%%%%%%%%%%%%%%%%%%%%%%%%%%%%%

\begin{frame}[fragile]
\frametitle{Codigo del ejemplo}



\begin{bytefield}[endianness=little,bitwidth=0.10\linewidth]{9}
	\bitheader[endianness=little,]{0-9} \\
	\colorbitbox{white}{1}{8} &
	\colorbitbox{lightgreen}{1}{4} &
	\colorbitbox{lightgreen}{1}{?} &
	\colorbitbox{lightgreen}{1}{?} &
	\colorbitbox{lightcyan}{2}{par1} &
	\colorbitbox{lightcyan}{2}{par2} &
	\colorbitbox{lightcyan}{1}{res} &
	\colorbitbox{lightred}{1}{527}
\end{bytefield}


\end{frame}
%%%%%%%%%%%%%%%%%%%%%%%%%%%%%%%%%%%%%%%%%%%%%%%%%%%%%%%%%%%%%%%%%%%%%%%%%%%%%%%%


\subsection{Displays}
\section{Nueva arquitectura para la máquina virtual}
\section{Paso de parámetros}
\section{Traducción}
\section{paso de parámetros: ampliación de la TS}

\end{document}
